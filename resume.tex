\documentclass{resume}
\usepackage{zh_CN-Adobefonts_external} 
\usepackage{linespacing_fix}
\usepackage{cite}
\begin{document}
\pagenumbering{gobble}



%***"%"后面的所有内容是注释而非代码,不会输出到最后的PDF中
%***使用本模板,只需要参照输出的PDF,在本文档的相应位置做简单替换即可
%***修改之后,输出更新后的PDF,只需要点击Overleaf中的“Recompile”按钮即可
%**********************************姓名********************************************
{\,\name{李\,皓}}

%**********************************联系信息****************************************
%第一个括号里写手机号,第二个写邮箱
\contactInfo{(+86) 18388417517}{qq376350262@163.com}
%**********************************其他信息****************************************
%在大括号内填写其他信息,最多填写4个,但是如果选择不填信息,
%那么大括号必须空着不写,而不能删除大括号。
%\otherInfo后面的四个大括号里的所有信息都会在一行输出
%如果想要写两行,那就用两次这个指令(\otherInfo{}{}{}{})即可

\otherInfo{出生年月:2002.1}{政治面貌:中共党员}{}{}

\otherInfo{个人主页:https://bxrjmfh.github.io/}{}{}{}
%*********************************照片**********************************************
%照片需要放到images文件夹下,名字必须是you.jpg,如果不需要照片可以不添加此行命令
%0.15的意思是,照片的宽度是页面宽度的0.15倍,调整大小,避免遮挡文字
\yourphoto{0.13}
%**********************************正文**********************************************


%***大标题,下面有横线做分割
%***一般的标题有:教育背景,实习(项目)经历,工作经历,自我评价,求职意向,等等
\section{教育经历}


%***********一行子标题**************
%***第一个大括号里的内容向左对齐,第二个大括号里的内容向右对齐
%***\textbf{}括号里的字是粗体,\textit{}括号里的字是斜体
\datedsubsection{\textbf{电子科技大学},计算机科学与工程学院,\textit{本科}}{2020.9 - 2024.6}
%***********列举*********************
%***可添加多个\item,得到多个列举项,类似的也可以用\textbf{}、\textit{}做强调
\begin{itemize} [parsep=1ex]
  \item \textit{就读专业:数据科学与大数据技术}
  
  \item \textit{前六学期均分:87.61,\quad{}GPA:3.89/4.0,\quad{}专业排名:6/31\quad{}}

  \item \textit{专业课程:C++程序设计 97;计算机组成原理 96;数据库原理与应用 95;\\人工智能 95;操作系统 94;计算机视觉(挑战课)91}
  
  \item \textit{外语水平:CET-4:562;CET-6:522}
\end{itemize}

\datedsubsection{\textbf{电子科技大学},计算机科学与工程学院,\textit{硕士研究生}}{2024.9 - 至今}
%***********列举*********************
%***可添加多个\item,得到多个列举项,类似的也可以用\textbf{}、\textit{}做强调
\begin{itemize} [parsep=1ex]
  \item \textit{就读专业:计算机技术}  
  \item \textit{所在团队:未来媒体中心; 导师: 王鹏,国家青年人才}  
\end{itemize}
\section{项目经历}

% \datedsubsection{\textbf{视觉感知与学习团队} ,URP 2021-2022 计划}{2022.1-2022.10}
% \begin{itemize}
% \item \textit{在吴洪副教授指导下进行科研训练,包含深度学习基本知识学习,文献阅读以及机器学习工具的使用。在此期间,我对于CNN, Transformer等网络架构有一定的了解。}
% \end{itemize}

% \datedsubsection{\textbf{NUS 计算思维课程}}{2022.12}
% \begin{itemize}[parsep=0.5ex]
%   \item \textit{参加新加坡国立大学开设的计算思维课程,学习编程和建模的计算思维,所在小队获得Top Project Prize 奖。在此期间锻炼了建模和抽象能力,锻炼了英语表达能力。}
% \end{itemize}

\datedsubsection{\textbf{开悟强化学习实验}}{2023.1}
\begin{itemize}[parsep=0.5ex]
  \item \textit{利用腾讯提供的王者荣耀强化学习平台开悟,对代码进行调优。针对模型中的reward系数和网络结构进行修改;在训练过程中对每批样本数量进行修改,对loss和学习率进行修改,有效提升了模型的效果。}
\end{itemize}

\datedsubsection{\textbf{斯坦福cs231n实验}}{2023.2}
\begin{itemize}[parsep=0.5ex]
  \item \textit{独立完成了斯坦福计算机视觉课程的实验。实验包含Dense,CNN等网络的numpy实现,包括基本组件的双向传播;使用pytorch实现网络结构,熟悉了深度学习工具库的使用。该实验不仅加深我对于数学知识的理解,还增强了代码能力。}
\end{itemize}

\datedsubsection{\textbf{基于大语言模型的具身智能任务优化}}{2023.7-2024.6}
\begin{itemize}[parsep=0.5ex]
  \item \textit{在学院王鹏教授的指导下,进行具身智能方向研究。在此期间阅读大量关于具身智能任务的论文,每周参与组会讨论,在SOTA基础上,利用R2R和REVEIRE数据集对llama2进行LoRa微调,进行模型优化,}
\end{itemize}

\section{获奖经历}
\datedsubsection{\textbf{电子科技大学研究生新生奖学金},一等奖}{2024.9}
\datedsubsection{\textbf{电子科技大学校级优秀毕业生}}{2024.6}
\datedsubsection{\textbf{全国大学生数学建模竞赛},国家级二等奖(队长)}{2022.9}
\datedsubsection{\textbf{国家励志奖学金、校级优秀学生奖学金}}{2022.9, 2023.9}
% \datedsubsection{\textbf{校级标兵学生奖学金}}{2021.9}
% \datedsubsection{\textbf{电子科技大学数学建模竞赛},校级三等奖}{2021.6}
% \datedsubsection{\textbf{电子科技大学第七届数学建模新生赛},校级一等奖}{2021.5}
% \datedsubsection{院级文体活动获奖若干}{-}

\section{自我评价}
\quad \quad 我本科就读于电子科技大学计算机科学与工程学院数据科学与大数据技术专业,研究生保研至未来媒体实验室攻读研究生。实验室实力雄厚,组内方向聚焦于大语言模型和多模态方向,包括微调和RAG技术。本人熟悉python,熟悉torch框架,熟练使用huggingface等相关开源库。具有较强的文献检索和阅读能力,能够跟进并实现LLM相关算法。
% \quad \quad 本人现就读于电子科技大学计算机科学与工程学院数据科学与大数据技术专业,努力学习,善于思考,对于新鲜事物有强烈的好奇。自从大一开始就参与数学建模培训,积累了丰富的数学建模和写作经历,能熟练Python进行计算和绘图,使用LaTeX 进行论文撰写。怀着对于计算机视觉领域的兴趣,我通过进入实验室和选修相关课程等途径加深了认识,具有一定阅读文献以及编码的能力。在学生工作上担任副班长,能够热心帮助别人。







\end{document}
